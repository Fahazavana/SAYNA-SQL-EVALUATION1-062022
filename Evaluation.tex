\documentclass[11]{article}
\usepackage[T1]{fontenc}
\usepackage[utf8]{inputenc}
\usepackage{lmodern}
\usepackage[a4paper,left=10mm,right=10mm,top=10mm,bottom=10mm]{geometry}

\usepackage[french]{babel}

\usepackage{minted}
\usepackage{xcolor}
\usemintedstyle{vim}
\definecolor{bg}{rgb}{0,0,0}
\begin{document}

\section{Analyse et correction de la base de données}

\subsection{Chargement et Correction apporté à Biblio\_base.mysql}

Ordonner l'ordre de la création des tables pour eviter les erreur
suivant :

\begin{itemize}
	\item Error Code: 1824. Failed to open the referenced table `oeuvres'
	\item Error Code: 1824. Failed to open the referenced table `adherents'
	\item Error Code: 1824. Failed to open the referenced table `livres'
\end{itemize}


Ainsi l'ordre de création des tables sera :
\begin{enumerate}
	\item oeuvres
	\item adherants
	\item livres
	\item emprunter
\end{enumerate}

\begin{minted}[bgcolor=bg]{mysql}
DROP DATABASE IF EXISTS biblio;
CREATE DATABASE biblio;

USE biblio;

CREATE TABLE oeuvres(
	NO 		integer primary key auto_increment,
	titre 		varchar(150) not null,
	auteur 		varchar(100),
	annee		integer,
	genre		varchar(30)
) ENGINE InnoDB;

CREATE TABLE adherents (
	NA		INT PRIMARY KEY AUTO_INCREMENT,
	nom		VARCHAR(30) not null,
	prenom		VARCHAR(30),
	adr		VARCHAR(100) not null,
	tel		CHAR(10)
) ENGINE InnoDB;

CREATE TABLE livres (
	NL		integer primary key auto_increment,
	editeur		varchar(50),
	NO		integer not null, foreign key(NO) references oeuvres(NO)
) ENGINE InnoDB;

CREATE TABLE emprunter (
	NL		integer not null, foreign key(NL) references livres(NL),
	dateEmp		date not null,
	dureeMax	integer not null,
	dateRet 	date,
	NA		integer not null, foreign key(NA) references adherents(NA),
	primary key (NL, dateEmp),
	index(dateEmp)
) ENGINE InnoDB;
\end{minted}

\begin{minted}[bgcolor=bg]{mysql}
INSERT INTO oeuvres VALUES 
	(1,'Narcisse et Goldmund','Hermann HESSE', 1930, 'Roman'),
	(2,'Bérénice','Jean RACINE', 1670, 'Théâtre'),
	(3,'Prolégomènes à  toute métaphysique future','Emmanuel KANT', 1783, 'Philosophie'),
	(4,'Mon coeur mis à nu','Charles BAUDELAIRE', 1887, 'Journal'),
	(5,'Voyage au bout de la nuit','Louis-Ferdinand CELINE', 1932, 'Roman'),
	(6,'Les possédés','Fedor DOSTOIEVSKI', 1872, 'Roman'),
	(7,'Le Rouge et le Noir','STENDHAL', 1830, 'Roman'),
	(8,'Alcibiade','Jacqueline de ROMILLY', 1995, 'Roman'),
	(9,'Monsieur Teste','Paul VALERY', 1926, 'Roman'),
	(10,'Lettres de Gourgounel','Kenneth WHITE', 1979, 'Récit'),
	(11,'Lettres à un jeune poète','Rainer Maria RILKE', 1929, 'Lettre'),
	(12,'Logique sans peine','Lewis CAROLL', 1887, 'Logique'),
	(13,'L''éthique','Baruch SPINOZA', 1677, 'Philosophie'),
	(14,'Sur le rêve','Sigmund FREUD', 1900, 'Philosophie'),
	(15,'Sens et dénotation','Gottlob FREGE', 1892, 'Philosophie'),
	(16,'Penser la logique','Gilbert HOTTOIS', 1989, 'Philosophie'),
	(17,'Au coeur des ténèbres','Joseph CONRAD',1899, 'Roman'),
	(18,'Jan Karski','Yannick HAENEL', 2009, 'Roman');	
\end{minted}

\begin{minted}[bgcolor=bg]{mysql}
INSERT INTO adherents VALUES 
	(1,'Lecoeur','Jeanette','16 rue de la République, 75010 Paris','0145279274'),
	(2,'Lecoeur','Philippe','16 rue de la République, 75010 Paris','0145279274'),
	(3,'Dupont','Yvan','73 rue Lamarck, 75018 Paris','0163538294'),
	(4,'Mercier','Claude','155 avenue Parmentier, 75011 Paris','0136482736'),
	(5,'Léger','Marc','90 avenue du Maine, 75014 Paris','0164832947'),
	(6,'Martin','Laure','51 boulevard Diderot, 75012 Paris','0174693277'),
	(7,'Crozier','Martine','88 rue des Portes Blanches, 75018 Paris','0146829384'),
	(8,'Lebon','Clément','196 boulevard de Sebastopol, 75001 Paris','0132884739'),
	(9,'Dufour','Jacques','32 rue des Alouettes, 75003 Paris','0155382940'),
	(10,'Dufour','Antoine','32 rue des Alouettes, 75003 Paris','0155382940'),
	(11,'Dufour','Stéphanie','32 rue des Alouettes, 75003 Paris','0155382940'),
	(12,'Raymond','Carole','35 rue Oberkampf, 75011 Paris','0153829402'),
	(13,'Durand','Albert','4 rue Mandar, 75002 Paris','0642374021'),
	(14,'Wilson','Paul','12 rue Paul Vaillant Couturier, 92400 Montrouge','0642327407'),
	(15,'Grecault','Philippe','15 rue de la Roquette, 75012 Paris','0132762983'),
	(16,'Carre','Stéphane','51 boulevard Diderot, 75012 Paris','0174693277'),
	(17,'Johnson','Astrid','3 rue Léon Blum, 75002 Paris','0143762947'),
	(18,'Mirou','Caroline','2 square Mirabeau, 75011 Paris','0163827399'),
	(19,'Espelette','Jean-Jacques','141 avenue de France, 75019 Paris','0134887264'),
	(20,'Despentes','Anthony','56 boulevard de la Villette, 75019 Paris','0133889463'),
	(21,'Terlu','Véronique','45 rue des Batignolles, 75020 Paris','0165998372'),
	(22,'Rihour','Cécile','7 rue Montorgueil, 75002 Paris','0166894754'),
	(23,'Franchet','Pierre','160 rue Montmartre, 75009 Paris','0633628549'),
	(24,'Trochet','Ernest','34 rue de l''Esperance, 75008 Paris','0638295563'),
	(25,'Gainard','Simon','55 rue Desnouettes, 75015 Paris','0174928934'),
	(26,'Touche','Johanna','14 rue du Bac, 75006 Paris','0619384065'),
	(27,'Cornu','Sylvain','22 rue Mouffetard, 75005 Paris','0184927489'),
	(28,'Frederic','Cyril','15 rue du Simplon, 75018 Paris','0173625492'),
	(29,'Crestard','Cedric','5 rue Doudeauville, 75018 Paris','0629485700'),
	(30,'Le Bihan','Karine','170 bis rue Ordener, 75018 Paris','0638995221');
\end{minted}
\newpage
\begin{minted}[bgcolor=bg]{mysql}
INSERT INTO livres VALUES
	(1,'GF',1),
	(2,'FOLIO',2),
	(3,'HACHETTE',3),
	(4,'GF',4),
	(5,'FOLIO',5),
	(6,'FOLIO',6),
	(7,'GF',7),
	(8,'FOLIO',8),
	(9,'HACHETTE',9),
	(10,'GF',10),
	(11,'HACHETTE',11),
	(12,'FOLIO',12),
	(13,'GF',13),
	(14,'FOLIO',14),
	(15,'HACHETTE',15),
	(16,'HACHETTE',16),
	(17,'GF',1),
	(18,'FOLIO',2),
	(19,'HACHETTE',2),
	(20,'FOLIO',4),
	(21,'GF',5),
	(22,'HACHETTE',4),
	(23,'HACHETTE',7),
	(24,'FOLIO',8),
	(25,'GF',1),
	(26,'HACHETTE',10),
	(27,'FOLIO',11),
	(28,'FOLIO',12),
	(29,'GF',1),
	(30,'HACHETTE',14),
	(31,'FOLIO',17),
	(32,'GALLIMARD',18);
\end{minted}
\newpage
\begin{minted}[bgcolor=bg]{mysql}
INSERT INTO emprunter VALUES
	(1,from_days(to_days(current_date)-350),21,from_days(to_days(current_date)-349),26),
	(4,from_days(to_days(current_date)-323),21,from_days(to_days(current_date)-310),4),
	(26,from_days(to_days(current_date)-315),21,from_days(to_days(current_date)-318),9),
	(25,from_days(to_days(current_date)-311),21,from_days(to_days(current_date)-293),1),
	(12,from_days(to_days(current_date)-300),21,from_days(to_days(current_date)-1290),7),
	(20,from_days(to_days(current_date)-283),21,from_days(to_days(current_date)-282),27),
	(10,from_days(to_days(current_date)-273),21,from_days(to_days(current_date)-250),7),
	(4,from_days(to_days(current_date)-232),14,from_days(to_days(current_date)-228),12),
	(24,from_days(to_days(current_date)-226),14,from_days(to_days(current_date)-220),26),
	(8,from_days(to_days(current_date)-201),14,from_days(to_days(current_date)-183),13),
	(6,from_days(to_days(current_date)-199),14,from_days(to_days(current_date)-194),3),
	(10,from_days(to_days(current_date)-169),14,from_days(to_days(current_date)-157),8),
	(1,from_days(to_days(current_date)-153),14,from_days(to_days(current_date)-142),3),
	(15,from_days(to_days(current_date)-146),14,from_days(to_days(current_date)-138),10),
	(1,from_days(to_days(current_date)-106),14,from_days(to_days(current_date)-101),2),
	(4,from_days(to_days(current_date)-103),14,from_days(to_days(current_date)-93),5),
	(18,from_days(to_days(current_date)-86),14,from_days(to_days(current_date)-79),3),
	(8,from_days(to_days(current_date)-76),14,from_days(to_days(current_date)-70),18),
	(2,from_days(to_days(current_date)-37),14,from_days(to_days(current_date)-28),4),
	(1,from_days(to_days(current_date)-28),14,from_days(to_days(current_date)-23),1),
	(3,from_days(to_days(current_date)-21),14,from_days(to_days(current_date)-17),3),
	(20,from_days(to_days(current_date)-24),14,from_days(to_days(current_date)-8),9),
	(21,from_days(to_days(current_date)-23),14,from_days(to_days(current_date)-11),14),
	(2,from_days(to_days(current_date)-10),14, NULL,28),
	(9,from_days(to_days(current_date)-10),14, NULL,28),
	(14,from_days(to_days(current_date)-9),14, NULL,1),
	(16,from_days(to_days(current_date)-9),14, NULL,1),
	(5,from_days(to_days(current_date)-5),14, NULL,16),
	(29,from_days(to_days(current_date)-395),14, NULL,27),
	(11,from_days(to_days(current_date)-30),14, NULL,22),
	(31,from_days(to_days(current_date)-1),14, NULL,20),
	(21,from_days(to_days(current_date)-1),14, NULL,20),
	(32,from_days(to_days(current_date)-1),14, NULL,20);
\end{minted}


\subsection{Nombres de tuples après l'execution du script:}
Pour la table oeuvres :
\begin{minted}[bgcolor=bg]{mysql}
SELECT COUNT(*) FROM `biblio`.`oeuvres`;
+----------+
| COUNT(*) |
+----------+
|       18 |
+----------+
1 row in set (0,04 sec)
\end{minted}
Pour la table adherents :
\begin{minted}[bgcolor=bg]{mysql}
SELECT COUNT(*) FROM `biblio`.`adherents`;
+----------+
| COUNT(*) |
+----------+
|       30 |
+----------+
1 row in set (0,04 sec)
\end{minted}
Pour la table livres :
\begin{minted}[bgcolor=bg]{mysql}
SELECT COUNT(*) FROM `biblio`.`livres`;

+----------+
| COUNT(*) |
+----------+
|       32 |
+----------+
1 row in set (0,01 sec)
\end{minted}

Pour la table emprunter :

\begin{minted}[bgcolor=bg]{mysql}
SELECT COUNT(*) FROM `biblio`.`emprunter`;
+----------+
| COUNT(*) |
+----------+
|       33 |
+----------+
1 row in set (0,05 sec)	
\end{minted}

\textbf{Ainsi pour obtenir le nombre totale de Tuple on fait :}

\begin{minted}[bgcolor=bg]{mysql}
SELECT 
	(SELECT COUNT(*) FROM `biblio`.`oeuvres`) 
	+
	(SELECT COUNT(*) FROM `biblio`.`adherents`)
	+
	(SELECT COUNT(*) FROM `biblio`.`livres`)
	+
	(SELECT COUNT(*) FROM `biblio`.`emprunter`)
	AS `Nombres de tuples`;

+-------------------+
| Nombres de tuples |
+-------------------+
|               113 |
+-------------------+
1 row in set (0,80 sec)
\end{minted}

\subsection{Nombre d'attribut :}
\begin{itemize}
	\item Pour la table oeuvres :
	      \begin{minted}[bgcolor=bg]{mysql}
DESCRIBE `biblio`.`oeuvres`;

+--------+--------------+------+-----+---------+----------------+
| Field  | Type         | Null | Key | Default | Extra          |
+--------+--------------+------+-----+---------+----------------+
| NO     | int          | NO   | PRI | NULL    | auto_increment |
| titre  | varchar(150) | NO   |     | NULL    |                |
| auteur | varchar(100) | YES  |     | NULL    |                |
| annee  | int          | YES  |     | NULL    |                |
| genre  | varchar(30)  | YES  |     | NULL    |                |
+--------+--------------+------+-----+---------+----------------+
5 rows in set (0,02 sec)
\end{minted}
	      On a donc 5 attributs pour la table oeuvre.

	\item Pour la table adherents :
	      \begin{minted}[bgcolor=bg]{mysql}
DESCRIBE `biblio`.`adherents`;

+--------+--------------+------+-----+---------+----------------+
| Field  | Type         | Null | Key | Default | Extra          |
+--------+--------------+------+-----+---------+----------------+
| NA     | int          | NO   | PRI | NULL    | auto_increment |
| nom    | varchar(30)  | NO   |     | NULL    |                |
| prenom | varchar(30)  | YES  |     | NULL    |                |
| adr    | varchar(100) | NO   |     | NULL    |                |
| tel    | char(10)     | YES  |     | NULL    |                |
+--------+--------------+------+-----+---------+----------------+
5 rows in set (0,00 sec)
\end{minted}

	      Donc on a 5 pour la tables adherents.
	\item Pour la table livres :
	      \begin{minted}[bgcolor=bg]{mysql}
DESCRIBE `biblio`.`livres`;

+---------+-------------+------+-----+---------+----------------+
| Field   | Type        | Null | Key | Default | Extra          |
+---------+-------------+------+-----+---------+----------------+
| NL      | int         | NO   | PRI | NULL    | auto_increment |
| editeur | varchar(50) | YES  |     | NULL    |                |
| NO      | int         | NO   | MUL | NULL    |                |
+---------+-------------+------+-----+---------+----------------+
3 rows in set (0,03 sec)
\end{minted}
	      Donc on a 3 attributs por la table livres.

	\item Pour la table emprunter :
	      \begin{minted}[bgcolor=bg]{mysql}
DESCRIBE `biblio`.`emprunter`;

+----------+------+------+-----+---------+-------+
| Field    | Type | Null | Key | Default | Extra |
+----------+------+------+-----+---------+-------+
| NL       | int  | NO   | PRI | NULL    |       |
| dateEmp  | date | NO   | PRI | NULL    |       |
| dureeMax | int  | NO   |     | NULL    |       |
| dateRet  | date | YES  |     | NULL    |       |
| NA       | int  | NO   | MUL | NULL    |       |
+----------+------+------+-----+---------+-------+
5 rows in set (0,07 sec)
\end{minted}
	      Donc on a 5 attributs pour la table emprunter
\end{itemize}
\subsection{Liste des clé Primaire}

Grace à la requette suivante on obtient les clé primaire:

\begin{minted}[bgcolor=bg]{mysql}
SELECT COLUMN_NAME AS PRIMARY_KEY
FROM INFORMATION_SCHEMA.COLUMNS
WHERE TABLE_SCHEMA = 'biblio'
AND TABLE_NAME = '<nom_table>' -- Nom de la tables
AND COLUMN_KEY = 'PRI';
\end{minted}

Pour la table oeuvres : NO

\begin{minted}[bgcolor=bg]{mysql}
+-------------+
| PRIMARY_KEY |
+-------------+
| NO          |
+-------------+
1 row in set (0,00 sec)
\end{minted}

Pour la table adherents : NA

\begin{minted}[bgcolor=bg]{mysql}
+-------------+
| PRIMARY_KEY |
+-------------+
| NA          |
+-------------+
1 row in set (0,00 sec)
\end{minted}

Pour la table livres : NL

\begin{minted}[bgcolor=bg]{mysql}
+-------------+
| PRIMARY_KEY |
+-------------+
| NL          |
+-------------+
1 row in set (0,00 sec)
\end{minted}

Pour la table emprunter : NL,dateEmp

\begin{minted}[bgcolor=bg]{mysql}
+-------------+
| PRIMARY_KEY |
+-------------+
| NL          |
| dateEmp     |
+-------------+
2 rows in set (0,00 sec)
\end{minted}

\newpage
\section{Intéraction avec la base de donné}

Avant de commencer on vas crée des vues pour faciliter les recherches et la manipulation de la base de donner.
\textbf{Vue Livres - Oeuvres} : permet de visualier les relation entres les deux tables
\begin{minted}[bgcolor=bg]{mysql}
CREATE 
	ALGORITHM = UNDEFINED 
	DEFINER = `root`@`localhost` 
	SQL SECURITY DEFINER
VIEW `Livres_Oeuvres` AS
	SELECT 
		`livres`.`NL` AS `NL`,
		`oeuvres`.`NO` AS `NO`,
		`oeuvres`.`titre` AS `titre`,
		`oeuvres`.`auteur` AS `auteur`,
		`livres`.`editeur` AS `editeur`,
		`oeuvres`.`annee` AS `annee`,
		`oeuvres`.`genre` AS `genre`
	FROM
		(`livres`
		JOIN `oeuvres` ON ((`livres`.`NO` = `oeuvres`.`NO`)))
	ORDER BY `livres`.`NL`;
\end{minted}
\textbf{Vue Emprunter - Adherent } : pour visualiser les adhérent qui on emprunter des livres
\begin{minted}[bgcolor=bg]{mysql}
CREATE 
    ALGORITHM = UNDEFINED 
    DEFINER = `root`@`localhost` 
    SQL SECURITY DEFINER
VIEW `Adherent_Emprunt` AS
SELECT 
    `emprunter`.`NL` AS `NL`,
    `emprunter`.`dateEmp` AS `dateEmp`,
    `emprunter`.`dureeMax` AS `dureeMax`,
    `emprunter`.`dateRet` AS `dateRet`,
    `adherents`.`NA` AS `NA`,
    `adherents`.`nom` AS `nom`,
    `adherents`.`prenom` AS `prenom`
FROM
    (`emprunter`
    JOIN `adherents`)
WHERE
    (`emprunter`.`NA` = `adherents`.`NA`)
ORDER BY `emprunter`.`NL` , `emprunter`.`dateEmp`
\end{minted}

\textbf{Vue Livre - Oeuvre - Adherent - Emprunt }: Pour une vue synthétisant les relation entres les 4  tables;
\begin{minted}[bgcolor=bg]{mysql}
CREATE 
    ALGORITHM = UNDEFINED 
    DEFINER = `root`@`localhost` 
    SQL SECURITY DEFINER
VIEW `Livre_Oeuvres_Adherent_Emprunt` AS
    SELECT 
        `Adherent_Emprunt`.`dateEmp` AS `dateEmp`,
        `Adherent_Emprunt`.`dateRet` AS `dateRet`,
        `Livres_Oeuvres`.`NL` AS `NL`,
        `Livres_Oeuvres`.`titre` AS `titre`,
		`Adherent_Emprunt`.`NA` AS `NA`,
        `Adherent_Emprunt`.`nom` AS `nom`,
        `Adherent_Emprunt`.`prenom` AS `prenom`
    FROM
        (`Livres_Oeuvres`
        JOIN `Adherent_Emprunt` ON ((`Livres_Oeuvres`.`NL` = `Adherent_Emprunt`.`NL`)))
ORDER BY `Adherent_Emprunt`.`dateEmp` , `Livres_Oeuvres`.`NL`
\end{minted}

\subsection{Quels sont les livres actuellement empruntés ?}

\begin{minted}[bgcolor=bg]{mysql}
SELECT 	DISTINCT(emprunter.NL),titre,auteur,editeur,annee 
		FROM Livres_Oeuvres 
		INNER JOIN emprunter 
		ON emprunter.NL=Livres_Oeuvres.NL 
ORDER BY NL;

+----+------------------------------------------+------------------------+-----------+-------+
| NL | titre                                    | auteur                 | editeur   | annee |
+----+------------------------------------------+------------------------+-----------+-------+
|  1 | Narcisse et Goldmund                     | Hermann HESSE          | GF        |  1930 |
|  2 | Bérénice                                 | Jean RACINE            | FOLIO     |  1670 |
|  3 | Prolégomènes à  toute métaphysique future| Emmanuel KANT          | HACHETTE  |  1783 |
|  4 | Mon coeur mis à nu                       | Charles BAUDELAIRE     | GF        |  1887 |
|  5 | Voyage au bout de la nuit                | Louis-Ferdinand CELINE | FOLIO     |  1932 |
|  6 | Les possédés                             | Fedor DOSTOIEVSKI      | FOLIO     |  1872 |
|  8 | Alcibiade                                | Jacqueline de ROMILLY  | FOLIO     |  1995 |
|  9 | Monsieur Teste                           | Paul VALERY            | HACHETTE  |  1926 |
| 10 | Lettres de Gourgounel                    | Kenneth WHITE          | GF        |  1979 |
| 11 | Lettres à un jeune poète                 | Rainer Maria RILKE     | HACHETTE  |  1929 |
| 12 | Logique sans peine                       | Lewis CAROLL           | FOLIO     |  1887 |
| 14 | Sur le rêve                              | Sigmund FREUD          | FOLIO     |  1900 |
| 15 | Sens et dénotation                       | Gottlob FREGE          | HACHETTE  |  1892 |
| 16 | Penser la logique                        | Gilbert HOTTOIS        | HACHETTE  |  1989 |
| 18 | Bérénice                                 | Jean RACINE            | FOLIO     |  1670 |
| 20 | Mon coeur mis à nu                       | Charles BAUDELAIRE     | FOLIO     |  1887 |
| 21 | Voyage au bout de la nuit                | Louis-Ferdinand CELINE | GF        |  1932 |
| 24 | Alcibiade                                | Jacqueline de ROMILLY  | FOLIO     |  1995 |
| 25 | Narcisse et Goldmund                     | Hermann HESSE          | GF        |  1930 |
| 26 | Lettres de Gourgounel                    | Kenneth WHITE          | HACHETTE  |  1979 |
| 29 | Narcisse et Goldmund                     | Hermann HESSE          | GF        |  1930 |
| 31 | Au coeur des ténèbres                    | Joseph CONRAD          | FOLIO     |  1899 |
| 32 | Jan Karski                               | Yannick HAENEL         | GALLIMARD |  2009 |
+----+------------------------------------------+------------------------+-----------+-------+
23 rows in set (0,00 sec)
\end{minted}

\subsection{Quels sont les livres empruntés par Jeannette Lecoeur ? Vérifier dans
	la réponse qu'il n'y a pas d'homonymes.}


\begin{minted}[bgcolor=bg]{mysql}
SELECT * FROM biblio.Livre_Oeuvres_Adherent_Emprunt
		WHERE nom='Lecoeur' and prenom='Jeannette';

Empty set (0,00 sec)
\end{minted}

Aucun Résultat;

Par contre avec : Jeanette Lecoeur
\begin{minted}[bgcolor=bg]{mysql}
SELECT * FROM biblio.Livre_Oeuvres_Adherent_Emprunt WHERE nom='Lecoeur' and prenom='Jeanette';

+------------+------------+----+----------------------+-----------------+---------+----------+
| dateEmp    | dateRet    | NL | titre                | auteur          | nom     | prenom   |
+------------+------------+----+----------------------+-----------------+---------+----------+
| 2021-08-20 | 2021-09-07 | 25 | Narcisse et Goldmund | Hermann HESSE   | Lecoeur | Jeanette |
| 2022-05-30 | 2022-06-04 |  1 | Narcisse et Goldmund | Hermann HESSE   | Lecoeur | Jeanette |
| 2022-06-18 | NULL       | 14 | Sur le rêve          | Sigmund FREUD   | Lecoeur | Jeanette |
| 2022-06-18 | NULL       | 16 | Penser la logique    | Gilbert HOTTOIS | Lecoeur | Jeanette |
+------------+------------+----+----------------------+-----------------+---------+----------+
4 rows in set (0,00 sec)
\end{minted}

\subsection{Quels sont tous les livres empruntés en septembre 2009}
\begin{minted}[bgcolor=bg]{mysql}
SELECT dateEmp,DISTINCT(NL),titre,auteur FROM biblio.Livre_Oeuvres_Adherent_Emprunt
WHERE dateEmp BETWEEN '2009-09-01' AND '2009-09-30';

Empty set (0,00 sec)
\end{minted}

\subsection{Tous les adhérents qui ont emprunté un livre de Fedor Dostoievski.}
\begin{minted}[bgcolor=bg]{mysql}
SELECT	dateEmp,NL,titre,auteur,NA,nom,prenom 
		FROM biblio.Livre_Oeuvres_Adherent_Emprunt 
		WHERE auteur='Fedor DOSTOIEVSKI';

+------------+----+----------------+-------------------+----+--------+--------+
| dateEmp    | NL | titre          | auteur            | NA | nom    | prenom |
+------------+----+----------------+-------------------+----+--------+--------+
| 2021-12-10 |  6 | Les possédés   | Fedor DOSTOIEVSKI |  3 | Dupont | Yvan   |
+------------+----+----------------+-------------------+----+--------+--------+
1 row in set (0,00 sec)
\end{minted}

\subsection{ Un nouvel adhérent vient de s’inscrire : Olivier DUPOND, 76, quai de la Loire,
	75019 Paris, téléphone : 0102030405}
\begin{minted}[bgcolor=bg]{mysql}
INSERT INTO `biblio`.`adherents` (`nom`, `prenom`, `adr`, `tel`) 
	VALUES ('Dupon', 'Olivier', '76 quai de la Loire, 75019 Paris', '0102030405');

Query OK, 1 row affected (0,18 sec)
\end{minted}


\subsection{Martine CROZIER vient d’emprunter « Au coeur des ténèbres » que vous venez
	d’ajouter et « Le rouge et le noir » chez Hachette, livre n°23. Faire les mises à
	jour de la BD.}

Pour Faire cela nos allons faire les requettes suivantes dans l'ordre
\begin{enumerate}

	\item  Chercher le NA de Martine CROZIER,

	      \begin{minted}[bgcolor=bg]{mysql}
SELECT * FROM adherents WHERE nom='Crozier' and prenom='Martine';

+----+---------+---------+-----------------------------------------+------------+
| NA | nom     | prenom  | adr                                     | tel        |
+----+---------+---------+-----------------------------------------+------------+
|  7 | Crozier | Martine | 88 rue des Portes Blanches, 75018 Paris | 0146829384 |
+----+---------+---------+-----------------------------------------+------------+
1 row in set (0,01 sec)
\end{minted}
	\item Trouver les deux Livres à emprunter
	      \begin{minted}[bgcolor=bg]{mysql}
SELECT * FROM biblio.Livres_Oeuvres
	WHERE titre='Au coeur des ténèbres' 
	OR 
	(titre='Le rouge et le noir' AND editeur='Hachette' AND NL=23);
	
+----+----+-------------------------+---------------+----------+-------+-------+
| NL | NO | titre                   | auteur        | editeur  | annee | genre |
+----+----+-------------------------+---------------+----------+-------+-------+
| 23 |  7 | Le Rouge et le Noir     | STENDHAL      | HACHETTE |  1830 | Roman |
| 31 | 17 | Au coeur des ténèbres   | Joseph CONRAD | FOLIO    |  1899 | Roman |
+----+----+-------------------------+---------------+----------+-------+-------+
2 rows in set (0,00 sec)
\end{minted}
	\item Insérér le NA de l'empruntuer, le NL des livres à emprunter,la date d'emprunt dateEmp et la durée maximum dureeMax d'emprunt
	      \begin{minted}[bgcolor=bg]{mysql}
INSERT INTO `biblio`.`emprunter` (`NL`, `dateEmp`, `dureeMax`, `NA`) 
	VALUES 
		('31', '2022-06-28', '14', '7'),
		('23', '2022-06-28', '14', '7');

	Query OK, 2 rows affected (0,17 sec)
	Records: 2  Duplicates: 0  Warnings: 0
\end{minted}
\end{enumerate}


\subsection{ M. Cyril FREDERIC ramène les livres qu’il a empruntés. Faire la mise à jour de
	la BD.}
\begin{enumerate}
	\item On cherch les livres qu'il a emprunter
	      \begin{minted}[bgcolor=bg]{mysql}
SELECT 
		*
FROM
	biblio.Livre_Oeuvres_Adherent_Emprunt
	WHERE 
		nom='Frederic' AND prenom='Cyril';

+------------+---------+----+----------------+-------------+----+----------+--------+
| dateEmp    | dateRet | NL | titre          | auteur      | NA | nom      | prenom |
+------------+---------+----+----------------+-------------+----+----------+--------+
| 2022-06-17 | NULL    |  2 | Bérénice       | Jean RACINE | 28 | Frederic | Cyril  |
| 2022-06-17 | NULL    |  9 | Monsieur Teste | Paul VALERY | 28 | Frederic | Cyril  |
+------------+---------+----+----------------+-------------+----+----------+--------+
2 rows in set (0,00 sec)
		\end{minted}
	      Ainsi on a les deux clé primire pour mettre à jour la table emprunt : (2,2022-06-17) et (9,2022-06-17)

	\item  On met à jour les deux enregitements correspondant avec dateRet='2022-06-28'
	      \begin{minted}[bgcolor=bg]{mysql}
UPDATE `biblio`.`emprunter` SET `dateRet` = '2022-06-28' 
	WHERE (`NL` = '2') and (`dateEmp` = '2022-06-17');

Query OK, 1 row affected (0,21 sec)
Rows matched: 1  Changed: 1  Warnings: 0

			

UPDATE `biblio`.`emprunter` SET `dateRet` = '2022-06-28' 
	WHERE (`NL` = '9') and (`dateEmp` = '2022-06-17');

Query OK, 1 row affected (0,13 sec)
Rows matched: 1  Changed: 1  Warnings: 0
\end{minted}
\end{enumerate}

\end{document}